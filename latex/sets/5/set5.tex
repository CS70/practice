% !TEX TS-program = pdflatex
% !TEX encoding = UTF-8 Unicode

% This is a simple template for a LaTeX document using the "article" class.
% See "book", "report", "letter" for other types of document.

\documentclass[11pt, preview]{standalone} % use larger type; default would be 10pt

\usepackage[utf8]{inputenc} % set input encoding (not needed with XeLaTeX)


\usepackage{../../markup}
%%% Examples of Article customizations
% These packages are optional, depending whether you want the features they provide.
% See the LaTeX Companion or other references for full information.

%%% PAGE DIMENSIONS
\usepackage{geometry} % to change the page dimensions
\geometry{a4paper} % or letterpaper (US) or a5paper or....
% \geometry{margin=2in} % for example, change the margins to 2 inches all round
% \geometry{landscape} % set up the page for landscape
%   read geometry.pdf for detailed page layout information

\usepackage{graphicx} % support the \includegraphics command and options
\usepackage{color}
% \usepackage[parfill]{parskip} % Activate to begin paragraphs with an empty line rather than an indent

%%% PACKAGES
\usepackage{amsmath, amsfonts,amssymb}
\usepackage{booktabs} % for much better looking tables
\usepackage{array} % for better arrays (eg matrices) in maths
\usepackage{paralist} % very flexible & customisable lists (eg. enumerate/itemize, etc.)
\usepackage{verbatim} % adds environment for commenting out blocks of text & for better verbatim
\usepackage{subfig} % make it possible to include more than one captioned figure/table in a single float
% These packages are all incorporated in the memoir class to one degree or another...

%%% HEADERS & FOOTERS
\usepackage{fancyhdr} % This should be set AFTER setting up the page geometry
\pagestyle{fancy} % options: empty , plain , fancy
\renewcommand{\headrulewidth}{0pt} % customise the layout...
\lhead{}\chead{}\rhead{}
\lfoot{}\cfoot{\thepage}\rfoot{}

%%% SECTION TITLE APPEARANCE
\usepackage{sectsty}
\allsectionsfont{\sffamily\mdseries\upshape} % (See the fntguide.pdf for font help)
% (This matches ConTeXt defaults)

%%% ToC (table of contents) APPEARANCE
\usepackage[nottoc,notlof,notlot]{tocbibind} % Put the bibliography in the ToC
\usepackage[titles,subfigure]{tocloft} % Alter the style of the Table of Contents
\renewcommand{\cftsecfont}{\rmfamily\mdseries\upshape}
\renewcommand{\cftsecpagefont}{\rmfamily\mdseries\upshape} % No bold!

\newcommand{\N}{\mathbb{N}}
\newcommand{\Z}{\mathbb{Z}}
\newcommand{\R}{\mathbb{R}}
\newcommand{\Q}{\mathbb{Q}}
%%% END Article customizations

%%% The "real" document content comes below...

%\date{} % Activate to display a given date or no date (if empty),
         % otherwise the current date is printed 

\begin{document}
\config{name}{CRT, RSA, Polynomial Interpolation, and Secret Sharing}
\noindent{\bf CRT}.

Consider the system:
$$x = 2 \ (\bmod \ 3)$$
$$x = 3 \ (\bmod \ 5)$$

Let $v_1$ be a magic number that is equivalent to $1 \ (\bmod \ 3)$ and  $0 \ (\bmod \ 5)$.
Let $v_2$ be a magic number that is equivalent to $0 \ (\bmod \ 3)$ and $1 \ (\bmod \ 5)$.

\begin{enumerate}
\item If we have some solution $x=y$, then $y+15$ is also a solution to the system.
	\begin{Choices} 
	\begin{itemize}
	\TrueChoice\item True
	\FalseChoice\item False
	\Solution True, since adding $15$ is the same as adding $0 \ (\bmod \ 3)$ and $(\bmod \ 5)$ respectively. This means that when we look for solutions to this system, we should look for a solution in $(\bmod \ 15)$
	\end{itemize}
	\end{Choices}
\item 
	We think that we can write $x= a \cdot v_1 + b \cdot v_2$. What should $a$ and $b$ be?
	\begin{enumerate}
		\item \begin{Freeform}{2}
			What is $a \ (\bmod \ 15)$?
			\Hint What is $v_1 + v_2 \ (\bmod \ 3)$? What is $v_1 + v_2 \ (\bmod \ 5)$? The numbers $v_1$ and $v_2$ are often likened to basis vectors.
			\Solution $x = 2 \cdot v_1+3 \cdot v_2$
			\end{Freeform}
		\item \begin{Freeform}{3}
			What is $b \ (\bmod \ 15)$?
			\Hint What is $v_1 + v_2 \ (\bmod \ 3)$? What is $v_1 + v_2 \ (\bmod \ 5)$? The numbers $v_1$ and $v_2$ are often likened to basis vectors.
			\Solution $x = 2 \cdot v_1+3 \cdot v_2$
			\end{Freeform}
	\end{enumerate}

\item \begin{Freeform}{10}
	What is $v_1$ in $(\bmod \ 15)$? 
	\Hint We need $v_1$ to be $0 \ (\bmod \ 5)$, so lets consider giving it a factor of $5$. We also need $v_1$ to be $(1 \bmod \ 3)$. Note that any multiple of $5$ will remain $0 \ (\bmod \ 5)$. What multiple should we choose such that the result is also $(1 \bmod \ 3)$?
	\Solution $v_1 = 5 \cdot (5^{-1} \bmod \ 3) = 5 \cdot 2 \Rightarrow v_1 = 10 \ (\bmod \ 15)$
	\end{Freeform}

\item \begin{Freeform}{6}
	What is $v_2$ in $(\bmod \ 15)$? 
	\Hint We need $v_2$ to be $0 \ (\bmod \ 3)$, so lets consider giving it a factor of $3$. We also need $v_2$ to be $(1 \bmod \ 5)$. Note that any multiple of $3$ will remain $0 \ (\bmod \ 3)$. What multiple should we choose such that the result is also $(1 \bmod \ 5)$?
	\Solution $v_2 = 3 \cdot (3^{-1} \bmod \ 5) = 3 \cdot 2 \Rightarrow v_2 = 6 \ (\bmod \ 15)$
	\end{Freeform}

\item \begin{Freeform}{8}
	Putting it all together, what is $x$ in $(\bmod \ 15)$? 
	\Hint Plug in the magic numbers from Parts (2) and (3) into Part (1)!
	\Solution $x = 2 \cdot v_1+3 \cdot v_2 = 2 \cdot 10 + 3 \cdot 6 = 20 + 18 =  8 \ (\bmod \ 15)$
	\end{Freeform}

\end{enumerate}

\noindent{\bf RSA}.

\noindent Bob would like to receive encrypted messages from Alice via RSA.
\begin{enumerate}
\item Bob chooses $p = 7$ and $q = 11$. His public key is $(N,e)$. 
\begin{enumerate}
\item \begin{Freeform}{77}
$N =$ 
\Hint What is $N$ a function of $p$ and $q$?
\Solution $N = pq = 77$
\end{Freeform}
\begin{Freeform}{60}
\item $e$ is relatively prime to the number:
\Hint The answer is not $N$. It is derived from Fermat's little theorem.
\Solution $e$ is relatively prime to $(p - 1)(q - 1) = 60$. This is because when the message $x^e$ is raised to the power of $d = e^{-1} \bmod 60$, we get $x^{ed} \bmod N = x^{k(p - 1)(q - 1) + 1} \bmod N = x(x^{k(p - 1)(q - 1)}) \bmod N = x \bmod N$ according to Fermat's little theorem.
\end{Freeform}
\item $e$ need not be prime itself, but what is the smallest prime number $e$ can be? Use this value for $e$ in all subsequent computations.
\begin{Freeform}{7}
%\Hint 
$ e = $
\Solution The smallest prime number that is coprime with $60$ is $7$, so $e = 7$
\end{Freeform}
\begin{Freeform}{1}
\item What is $gcd(e,(p-1)(q-1))$?
\Hint Related to part (b)
\Solution $e$ is required to be coprime to $(p - 1)(q - 1)$, which means their gcd is $1$ by definition
\end{Freeform}
\item What is the decryption exponent $d$?
\begin{Freeform}{43}
$d = $
\Hint Recall that $d = e^{-1} \mod (p-1)(q-1)$.
\Solution To find $d$, we need to compute $e^{-1} \bmod 60$. Recall that $e = 7$ from part c). We can find the multiplicative inverse of $7 \bmod 60$ using the egcd algorithm, which gives us $43$.  
\end{Freeform}
\item Now imagine that Alice wants to send Bob the message $30$. She applies her encryption function $E$ to $30$. What is her encrypted message?
\begin{Freeform}{2}
$E(x) = $
\Hint Recall that $E(x) \equiv x^e \mod N$.
\Solution Alice uses the public key $(77,\ 7)$ to encode her message as $30^{7} \bmod 77$. We can compute this value using a combination of the Chinese Remainder Theorem and Fermat's Little Theorem.\\

Let $y = 30^7 \bmod 77$. We first use Chinese remainder theorem to express $y \bmod 7$ and $y \bmod 11$:

$$y\ \equiv\ a \bmod 7$$
$$y\ \equiv\ b \bmod 11$$

We can solve for $a$ and $b$ using Fermat's Little Theorem:

\begin{align*}
a\ &=\ 30^7 \bmod 7\\
&=\ 30 \times (30^6) \bmod 7\\
&=\ 30 \bmod 7 \text{, because $x^6 \bmod 7 = 1$ by FLT}\\ 
&=\ 2
\end{align*}

\begin{align*}
b\ &=\ 30^7 \bmod 11\\
&=\ (30 \bmod 11)^7 \bmod 11\\
&=\ 8^7 \bmod 11\\
&=\ 2 \times 2^{10} \times 2^{10} \bmod 11\\
&=\ 2 \text{, because $x^{10} \bmod 11 = 1$ by FLT}
\end{align*}

Now that we have $y \bmod 7$ and $y \bmod 11$, we can write a system of equations and solve algebraically for $y$. Because $y \equiv 2 \bmod 7$, we can express $y$ as $7s + 2,\ s\thinspace \in \mathbb{Z}$. Using the fact that $y \equiv 2 \bmod 11$, we have 

$$7s + 2\ \equiv\ 2 \bmod 11 \implies s\ \equiv\ 0 \bmod 11$$

From this we know that $s = 11t,\thinspace t\in \mathbb{Z}$. Substituting into the original equation, we have that

$$y\ =\ 7(11t) + 2\ =\ 77t + 2,\ t \in \mathbb{Z}$$

meaning $30^7 \bmod 77\ =\ 2$, so the message Alice sends is $\hat{x} = 2$.

\end{Freeform}
\item Bob receives the encrypted message, and applies his decryption function $D$ to it.
\begin{Freeform}{30}
$D(x) = $
\Hint Recall that $D(x) \equiv x^d \mod N$.
\Solution Bob applies his private key $d = 43$ to decode Alice's message. He has to compute 

$$2^{43}\thinspace \bmod 77$$

From the previous part, we know that $2\ =\ 30^7 \bmod 77$, giving 

$$(30^7)^{43}\thinspace \bmod 77\ =\ 30^{1 \bmod 60}\thinspace \bmod 77\ =\ 30$$

Here we applied Fermat's Little Theorem and the fact that $ed \equiv 1 \bmod (p - 1)(q - 1)$ to verify the answer.

\end{Freeform}
\end{enumerate}


\item Decide whether each of the following statements are true or false.
\begin{enumerate}
\item Bob has to publish his key $(N,e)$ to receive encrypted messages from Alice. 
\begin{Choices}
\begin{itemize}
\TrueChoice\item True 
\FalseChoice\item False
\Solution This is the unique feature of asymmetric or public key cryptography: the key $(N, e)$ is open information to everyone.
\end{itemize}
\end{Choices}
\item Eve needs to know Bob's key $d$ in order to send him encrypted messages.
\begin{Choices} 
\begin{itemize}
\FalseChoice\item True
\TrueChoice\item False
\Solution No, she only needs to know $(N, e)$ to encode the message.
\end{itemize}
\end{Choices}
\item The security of RSA relies on the computational intractability of determining $x$ from $y = x^e \mod N$, even when $y$, $e$, and $N$ are all known.
\begin{Choices} 
\begin{itemize}
\TrueChoice\item True
\FalseChoice\item False
\Solution True, the most efficient algorithms we know of to solve this problem are far too slow to crack RSA for extremely large primes.
\end{itemize}
\end{Choices}
\item $E(x) = x^e \mod N$ is a bijection on numbers $\mod N$.
\begin{Choices} 
\begin{itemize}
\TrueChoice\item True
\FalseChoice\item False
\Hint We say a function $M$ is a bijection from set $A$ to set $B$ iff every element of $A$ maps to a unique element of $B$. 
\Solution This is true. If $0 \leq x \leq N - 1$, then each value of $x$ will map to a unique value $x^e$ in that same set. 

\end{itemize}
\end{Choices}
\end{enumerate}

\noindent{\bf Polynomial Interpolation}. 

\item Three points uniquely determine a degree $2$ polynomial. Given the three points $\{(x_1,y_1) = (-1,2), (x_2,y_2) = (1,-2),(x_3,y_3) = (2,5)\}$ we wish to find the unique polynomial $p(x) = a_2 x^2 + a_1 x + a_0$ such that $p(x_i) = y_i$.  In this question we will find $p(x)$ by solving a system of linear equations:
\begin{enumerate}
\item Compute $a,b,c,d$ such that: $p(-1) =  a_2 \cdot a +  a_1  \cdot b + a_0  \cdot c = d$
\begin{Freeform}{1}
$a = $
\Solution $a = 1$. See solution to $d$ for explanation.
\end{Freeform}
\begin{Freeform}{-1}
$b = $
\Solution $b = -1$. See solution to $d$ for explanation.
\end{Freeform}
\begin{Freeform}{1}
$c = $
\Solution $c = 1$. See solution to $d$ for explanation.
\end{Freeform}
\begin{Freeform}{2}
$d = $
\Solution From the point $(-1, 2)$, we know that

$$p(-1)\ =\ a_2 \times (-1)^2\ +\ a_1 \times (-1)\ +\ a_0\ =\ 2$$

giving us $a = 1,\ b = -1,\ c = 1,\ d = 2$
\end{Freeform}

\item Compute $a,b,c,d$ such that: $p(1) = a \cdot a_2 + b \cdot a_1 + c \cdot a_0 = d$
\begin{Freeform}{1}
$a = $
\Solution $a = 1$. See solution to $d$ for explanation.
\end{Freeform}
\begin{Freeform}{1}
$b = $
\Solution $b = 1$. See solution to $d$ for explanation.
\end{Freeform}
\begin{Freeform}{1}
$c = $
\Solution $c = 1$. See solution to $d$ for explanation.
\end{Freeform}
\begin{Freeform}{-2}
$d = $
% \Hint Recall that $p(x_2) = y_2$.
\Solution From the point $(1, -2)$, we know that 

$$a_2 \times 1^2\ +\ a_1 \times 1\ +\ a_0\ =\ -2$$

giving us $a = 1,\ b = 1,\ c = 1,\ d = -2$
\end{Freeform}

\item Compute $a,b,c,d$ such that: $p(2) = a \cdot a_2 + b \cdot a_1 + c \cdot a_0 = d$
\begin{Freeform}{4}
$a = $
\Solution $a = 4$. See solution to $d$ for explanation.
\end{Freeform}
\begin{Freeform}{2}
$b = $
\Solution $b = 2$. See solution to $d$ for explanation.
\end{Freeform}
\begin{Freeform}{1}
$c = $
\Solution $c = 1$. See solution to $d$ for explanation.
\end{Freeform}
\begin{Freeform}{5}
$d = $
% \Hint Recall that $p(x_3) = y_3$.
\Solution From the point $(2, 5)$, we know that 

$$a_2 \times 2^2\ +\ a_1 \times 2\ +\ a_0\ =\ 5$$

giving us $a = 4,\ b = 2,\ c = 1,\ d = 5$
\end{Freeform}

\item Subtract polynomials $p(x_1)$ and $p(x_2)$ to determine the value for $a_1$:
\begin{Freeform}{-2}
$a_1 = $
\Solution We first compute $p(-1) - p(1)$:

\begin{align*}
a_2\ -\ &a_1\ +\ a_0\ = 2\\
-(a_2\ +\ &a_1\ +\ a_0\ = -2) 
\end{align*}

Giving us $-2a_1 = 4 \implies a_1 = -2$
\end{Freeform}

\item Solve the remaining system of two equations and two variables to determine $a_2$ and $a_0$:
\begin{Freeform}{3}
$a_2 = $ 
\Solution $a_2 = 3$. See solution to $a_0$ for explanation.
\end{Freeform}
\begin{Freeform}{-3}
$a_0 = $ 

\Solution We first substitute $a_1 = -2$ into two of the equations in the original system, giving

\begin{align*}
a_2\ +\ (-2) \times 1\ +\ a_0\ =\ -2 &\implies a_2\ +\ a_0\ =\ 0\\
4a_2\ +\ (-2) \times 2\ +\ a_0\ =\ 5 &\implies 4a_2\ +\ a_0\ =\ 9
\end{align*}

We can substitute $a_0 = -a_2$ into the second equation to get 

$$4a_2\ +\ a_0\ =\ 4a_2 - a_2\ =\ 9\ \implies\ a_2\ =\ 3$$

Because $a_0 = -a_2$,

$$a_0\ =\ -3$$
\end{Freeform}

\end{enumerate}

\item In this question we will find $p(x)$ using the Lagrange interpolation method. Recall from question 3 that we are given the three points $\{(x_1,y_1) = (-1,2), (x_2,y_2) = (1,-2),(x_3,y_3) = (2,5)\}$ we wish to find the unique polynomial $p(x) = a_2 x^2 + a_1 x + a_0$ such that $p(x_i) = y_i$. 
\begin{enumerate}
\item Compute $a,b,c$ such that $\Delta_1(x) = a(x-b)(x-c)$. Note $b,c$ are integers such that $b<c$:

 \begin{Freeform}{1/6}
 $a = $
 \Hint Recall that $\Delta_j(x)$ is $1$ for $x = x_j$ and $0$ for $x = x_i,\ i \not= j$
 \Solution $a = \frac{1}{6}$. See solution to $c$ for explanation.
 \end{Freeform}
 \begin{Freeform}{1}
 $b = $
 \Solution $b = 1$. See solution to $c$ for explanation.
 \end{Freeform}
 \begin{Freeform}{2}
 $c = $

 \Solution We know that $\Delta_1(x)$ should be $1$ for $x = x_1 = -1$ and $0$ for $x = x_2 = 1$ and $x = x_3 = 2$. Let's take care of the second condition first.

 One polynomial that is $0$ for $x = 1$ and $x = 2$ is 

 $$(x - 1)(x - 2)$$

 This gives us 

 $$b = 1$$
 $$c = 2$$ 

 Now we simply need to find the right scaling factor $a$ to make sure that $a(x - 1)(x - 2) = 1$ when $x = -1$. Plugging in $-1$, we see we need to satisfy

 $$a \times (-2) \times (-3)\ =\ 1\ \implies\ a = \frac{1}{6}$$
 \end{Freeform}
  
\item Compute $a,b,c$ such that $\Delta_2(x) = a x^2 + b x + c$:
 
 \begin{Freeform}{-1/2}
 $a = $
 \Hint Recall that $\Delta_j(x)$ is $1$ for $x = x_j$ and $0$ for $x = x_i,\ i \not= j$
 \Solution $a = -\frac{1}{2}$. See solution to $c$ for explanation.
 \end{Freeform}
 \begin{Freeform}{1/2}
 $b = $
 \Solution $b = \frac{1}{2}$. See solution to $c$ for explanation.
 \end{Freeform}
 \begin{Freeform}{1}
 $c = $

 \Solution Following the same procedure as above, let us first find a polynomial that is $0$ when $x = x_1 = -1$ and $x = x_3 = 2$. This gives us 

 $$(x + 1)(x - 2)$$

 To ensure that it will be $1$ when $x = x_2 = 1$, we scale by $\frac{1}{(1 + 1)(1 - 2)} = -\frac{1}{2}$, giving 

 $$-\frac{1}{2}(x + 1)(x - 2)$$

 Expanding, we get 

 $$-\frac{1}{2}x^2 + \frac{1}{2}x + 1$$

 meaning $a = -\frac{1}{2}$, $b = \frac{1}{2}$, and $c = 1$

 \end{Freeform}

\item Compute $a,b,c$ such that $\Delta_3(x) = a(x-b)(x-c)$. Note $b, c$ are integers such that $b < c$.
 
 \begin{Freeform}{1/3}
 $a = $
 \Hint Recall that $\Delta_j(x)$ is $1$ for $x = x_j$ and $0$ for $x = x_i,\ i \not= j$. This should be a fraction.
 \Solution $a = \frac{1}{3}$. See solution to $c$ for explanation.
 \end{Freeform}
 \begin{Freeform}{-1}
 $b = $
 \Solution $b = -1$. See solution to $c$ for explanation.
 \end{Freeform}
 \begin{Freeform}{1}
 $c = $

 \Solution Following the same procedure as above, let us first find a polynomial that is $0$ when $x = x_1 = -1$ and $x = x_2 = 1$. This gives us 

 $$(x + 1)(x - 1)$$

 To ensure that it will be $1$ when $x = x_2 = 2$, we scale by $\frac{1}{(2 + 1)(2 - 1)} = \frac{1}{3}$, giving 

 $$\frac{1}{3}(x + 1)(x - 1)$$

 meaning $a = \frac{1}{3}$, $b = -1$, and $c = 1$

 \end{Freeform}

\item Compute $a,b,c$ such that $y_1\Delta_1(x) + y_3\Delta_3(x) = a x^2+b x+c$.
\begin{Freeform}{2}
$a = $ %\Hint Recall...
\Solution $a = 2$. See solution to $c$ for explanation.
\end{Freeform}
\begin{Freeform}{-1}
$b = $ %\Hint Recall...
\Solution $b = -1$. See solution to $c$ for explanation.
\end{Freeform}
\begin{Freeform}{-1}
$c = $ %\Hint Recall...

\Solution Recall $y_1\ =\ 2$ and $y_3\ =\ 5$. Plugging those into the equation given, we have 

$$y_1 \Delta_1(x)\ +\ y_3\Delta_3(x)\ =\ 2 \left(\frac{1}{6} (x - 1) (x - 2)\right)\ +\ 5 \left(\frac{1}{3}(x + 1)(x - 1)\right) $$

Simplifying, we get 

$$\frac{1}{3} x^2\ -\ x\ +\ \frac{2}{3}\ +\ \frac{5}{3} x^2\ -\ \frac{5}{3}\ =\ 2x^2\ -\ x\ -\ 1$$

Meaning $a\, =\, 2$, $b\, =\, -1$, and $c\, =\, -1$

\end{Freeform}

\item Compute $a,b,c$ such that $y_2\Delta_2(x) = a x^2 + b x + c$.
\begin{Freeform}{1}
$a = $ %\Hint Recall...
\Solution $a = 1$. See solution to $c$ for explanation.
\end{Freeform}
\begin{Freeform}{-1}
$b = $ %\Hint Recall...
\Solution $b = -1$. See solution to $c$ for explanation.
\end{Freeform}
\begin{Freeform}{-2}
$c = $ %\Hint Recall...
\Solution Recall $y_2\, =\, -2$. Plugging this into the expression given, 

$$y_2 \Delta_2(x)\ =\ -2\left(-\frac{1}{2} x^2\ -\ \frac{1}{2} x\ -\ 1\right)\ =\ x^2\ -\ x\ -\ 2$$

meaning $a\, =\, 1$, $b\, =\, -1$, and $c\, =\, -2$

\end{Freeform}

\item Finally, compute $a_2, a_1, a_0$ such that $p(x) = a_2 x^2 + a_1 x + a_0 = \sum_{i = 1}^{3}{y_i \Delta_i(x)}$.
\begin{Freeform}{3}
$a_2 =$
\Solution $a_2 = 3$. See solution to $a_0$ for explanation.
\end{Freeform}
\begin{Freeform}{-2}
$a_1 =$
\Solution $a_1 = -2$. See solution to $a_0$ for explanation.
\end{Freeform}
\begin{Freeform}{-3}
$a_0 =$

\Solution We simply have to combine the two expressions we found for $y_1 \Delta_1(x)\ +\ y_3 \Delta_3(x)$ and $y_2 \Delta_2(x)$:

$$\sum_{i = 1}^{3}\, y_i \Delta_i(x)\ =\ 2x^2\ -\ x\ -\ 1\ +\ x^2\ -\ x\ -\ 2\ =\ 3 x^2\ -\ 2x\ -\ 3$$

This gives us $a_3\, =\, 3$, $a_2\, =\, -2$, and $a_1\, =\, -3$
\end{Freeform}

\end{enumerate}

\item In this question, we want to demonstrate the intuition behind the Lagrange interpolation technique.

Let $p(x)$ be a polynomial of degree 2 over GF(7). Suppose $p(1) = 2$, $p(2) = 1$ and $p(3) = 4$. We would like to find the coefficient representation for $p$.
\begin{enumerate}
\item Suppose we had polynomials, $p_1$, $p_2$, and $p_3$, of degree $2$  satisfying the following properties:
\begin{align*}
    p_1(1) = 1, \;\;p_1(2) = 0,\;\;p_1(3)=0\\
    p_2(1) = 0, \;\;p_2(2) = 1,\;\;p_2(3)=0\\
    p_3(1) = 0, \;\;p_3(2) = 0,\;\;p_3(3)=1\\
\end{align*}
We want to express $p$ in terms of $p_1,\;p_2$, and $p_3$ as $p = A \cdot p_1 + B \cdot p_2 + C \cdot p_3$. Compute $A, B, C$.
\begin{Freeform}{2}
$A=$
\end{Freeform}
\begin{Freeform}{1}
$B=$
\end{Freeform}
\begin{Freeform}{4}
$C=$
\end{Freeform}
\item Now let's actually find the coefficient representation of $p_1$. To start off with, what for must $p_1$ for some constant $c \in GF(7)$?
\begin{Choices}
\begin{itemize}
\FalseChoice \item $c(x-x_1)(x-x_2)$
\TrueChoice \item $c(x-x_2)(x-x_3)$
\FalseChoice \item $c(x-x_1)(x-x_3)$
\end{itemize}
\end{Choices}

\item What is the value of $c$?
\begin{Freeform}{4}
$c = $
\Solution We must multiply $(x-2)(x-3)$ by a constant factor that will make it equalled to $1$ at $x=1$. Thus we must take the multiplicative inverse of $(1-2)(1-3) = 2 \mod 7$. This turns out to be $4$. Thus, $c = 4$, and expanding shows that the coefficient representation of $p_1$ is $4x^2+x+3$.
\end{Freeform}

\item What are $a_1, b_1, c_1$ such that $p_1 = a_1x^2 + b_1x + c_1$?
\begin{Freeform}{4}
$a_1=$
\end{Freeform}
\begin{Freeform}{1}
$b_1=$
\end{Freeform}
\begin{Freeform}{3}
$c_1=$
\end{Freeform}

\item Now find $p_2 = a_2x^2 + b_2x + c_2$ and $p_3 = a_3x^2 + b_3x + c_3$ using similar method.
\begin{Freeform}{6}
$a_2=$
\end{Freeform}
\begin{Freeform}{4}
$b_2=$
\end{Freeform}
\begin{Freeform}{4}
$c_2=$
\end{Freeform}

\begin{Freeform}{4}
$a_3=$
\end{Freeform}
\begin{Freeform}{2}
$b_3=$
\end{Freeform}
\begin{Freeform}{1}
$c_3=$
\end{Freeform}

\item Using what we've done so far, find $a, b, c$ such that $p = ax^2 + bx + c$.
\begin{Freeform}{2}
$a=$
\end{Freeform}
\begin{Freeform}{0}
$b=$
\end{Freeform}
\begin{Freeform}{0}
$c=$
\end{Freeform}
\end{enumerate}

\item We've learned in class that there is a bijection between the coefficient representation and the value representation of a polynomial.
In this problem, we will find the transformation matrices of the bijection.
Consider a polynomial of the form $f(x) = ax^2+bx +c$ for $a,b,c\in GF(7)$.
Define the coefficient representation of $f$ as $[a, b, c]^T$. Define the value representation of $f$ as $[f(1), f(2), f(3)]^T$.
\begin{enumerate}
\item What is the coefficient representation of $x^2$, $x$ and $1$ respectively?
  Find $a, b, c$ such that the coordinate representation of $x^2$ is $[a, b, c]^T$.
 \begin{Freeform}{1}
  $a = $
 \end{Freeform}
 \begin{Freeform}{0}
  $b = $
 \end{Freeform}
 \begin{Freeform}{0}
  $c = $
 \end{Freeform}
    Find $a, b, c$ such that the coordinate representation of $x$ is $[a, b, c]^T$.
 \begin{Freeform}{0}
  $a = $
 \end{Freeform}
 \begin{Freeform}{1}
  $b = $
 \end{Freeform}
 \begin{Freeform}{0}
  $c = $
 \end{Freeform}
    Find $a, b, c$ such that the coordinate representation of $1$ is $[a, b, c]^T$.
 \begin{Freeform}{0}
  $a = $
 \end{Freeform}
 \begin{Freeform}{0}
  $b = $
 \end{Freeform}
  \begin{Freeform}{1}
  $c = $
  \Solution $x^2 = 1 \cdot x^2 + 0 \cdot x + 0 \cdot 1$, so the coefficient representation of $x^2$ is $[1, 0, 0]^T$. It is the first basis vector.
$x = 0 \cdot x^2 + 1 \cdot x + 0 \cdot 1$, so the coefficient representation of $x$ is $[0, 1, 0]^T$. It is the second basis vector.
$1 = 0 \cdot x^2 + 0 \cdot x + 1 \cdot 1$, so the coefficient representation of $1$ is $[0, 0, 1]^T$. It is the third basis vector.
 \end{Freeform}
\item To change from coefficient representation to value representation, we need to find the transformation matrix $A$. Recall from linear algebra that the first column of $A$ is the vector that the first basis vector gets mapped to by the transformation. What is the first column of $A$?
In other words, what is the value representation of the polynomial that corresponds to the first basis vector $[1, 0, 0]$ in coordinate representation?
\begin{Freeform}{1}
 Find $a, b, c$ such that the first column of $A$ is $[a, b, c]^T$. \\
  $a = $
 \end{Freeform}
 \begin{Freeform}{4}
  $b = $
 \end{Freeform}
 \begin{Freeform}{2}
  $c = $
  \Solution From the previous part, $x^2$ is the polynomial that corresponds to the first basis vector in coefficient representation.
  The value representation of $x^2$ is $[1^2, 2^2, 3^2]^T = [1, 4, 2]^T$ by definition.
 \end{Freeform}
\item  Using similar method, find the second and third column of $A$.
\begin{Freeform}{1}
 Find $a, b, c$ such that the second column of $A$ is $[a, b, c]^T$. \\
  $a = $
 \end{Freeform}
 \begin{Freeform}{2}
  $b = $
 \end{Freeform}
 \begin{Freeform}{3}
  $c = $
 \end{Freeform}
 \begin{Freeform}{1}
 Find $a, b, c$ such that the third column of $A$ is $[a, b, c]^T$. \\
  $a = $
 \end{Freeform}
 \begin{Freeform}{1}
  $b = $
 \end{Freeform}
 \begin{Freeform}{1}
  $c = $
 \end{Freeform}
 \item Now we shift attention to the value representation. What is the polynomial $f_1$ that corresponds to the first basis vector $[1, 0, 0]^T$ in value representation?
  Also find the polynomials $f_2, f_3$ that correspond to the second and third basis vector respectively. (Hint: $p_1, p_2, p_3$ we found in the last problem may be relevant here.)

 \begin{Freeform}{4}
  $f_1 = ax^2 + bx + c$ \\
  $a = $
 \end{Freeform}
 \begin{Freeform}{1}
  $b = $
 \end{Freeform}
 \begin{Freeform}{3}
  $c = $
 \end{Freeform}
  $f_2 = ax^2 + bx + c$
 \begin{Freeform}{6}
  $a = $
 \end{Freeform}
 \begin{Freeform}{4}
  $b = $
 \end{Freeform}
 \begin{Freeform}{4}
  $c = $
 \end{Freeform}
  $f_3 = ax^2 + bx + c$
 \begin{Freeform}{4}
  $a = $
 \end{Freeform}
 \begin{Freeform}{2}
  $b = $
 \end{Freeform}
 \begin{Freeform}{1}
  $c = $
  \Solution By definition of value representation, the polynomial $f_1$ that corresponds to the first basis vector $[1, 0, 0]^T$
  in value representation satisfies $f_1(1) = 1$, $f_1(2) = 0$ and $f_1(3) = 0$. This exactly the basis polynomial $p_1$ we found
  in Lagrange interpolation in the previous problem.

  Similarly, the polynomials that correspond to the second and third basis vector are $p_2$ and $p_3$ respectively.
 \end{Freeform}
 \item Now we find the transformation matrix $B$ from value representation to coefficient representation.
     Find $a, b, c, d, e, f, g, h, i$ such that

\[
 \begin{bmatrix}
    a&d&g\\
    b&e&h\\
    c&f&i\\
\end{bmatrix}
\]
 \begin{Freeform}{4}
  $a=$
 \end{Freeform}
  \begin{Freeform}{1}
  $b=$
 \end{Freeform}
  \begin{Freeform}{3}
  $c=$
 \end{Freeform}
  \begin{Freeform}{6}
  $d=$
 \end{Freeform}
  \begin{Freeform}{4}
  $e=$
 \end{Freeform}
  \begin{Freeform}{4}
  $f=$
 \end{Freeform}
  \begin{Freeform}{4}
  $g=$
 \end{Freeform}
  \begin{Freeform}{2}
  $h=$
 \end{Freeform}
  \begin{Freeform}{1}
  $i=$
   \Solution Similar to part (b), the $i$th column of $B$ is the coefficient representation of $f_i = p_i$.
 \end{Freeform}

 \item What is the relationship between matrix $B$, the transformation matrix from value representation to coefficient
 representation, and matrix $A$, the transformation matrix from coefficient representation to value representation?
 \begin{Choices}
 \begin{itemize}
 \FalseChoice \item $B = A^{-1}$
 \TrueChoice \item $B = A$
 \FalseChoice \item $B = 2A$
 \end{itemize}
 \end{Choices}

\end{enumerate}

\noindent{\bf Secret Sharing}.
\item Suppose you are in charge of setting up a secret sharing scheme where you want to distribute $n = 5$ shares to $5$ people such that any $k = 3$ or more people can figure out the secret, but two or fewer cannot. Suppose we are working over $GF(7)$.
\begin{enumerate}
\begin{Freeform}{2}
\item What is the degree of the polynomial you will use to distribute the shares? 
 \Hint How many points uniquely determine a degree $n$ polynomial?

 \Solution We want $3$ points to uniquely determine this polynomial, so it should be of degree $2$.
 \end{Freeform}

 \item You randomly choose the polynomial: $P(x) = 5x^2 + 3x + 3$. What is the secret?
 \begin{Freeform}{3}
 $P(0) = $
 \Hint Remember we're working in $GF(7)$.
 \Solution The y-intercept, $3$, is our secret.
 \end{Freeform}
 
 \item What is the share given to the first official?
 \begin{Freeform}{4}
 $P(1) = $ 
 % \Hint Remember we're working in $GF(7)$.
 \Solution $$P(1)\ =\ (5\ +\ 3\ +\ 3) \bmod 7\ =\ 11 \bmod 7\ =\ 4$$
 \end{Freeform}
 
 \item What is the share given to the second official?
 \begin{Freeform}{1}
 % \Hint Remember we're working in $GF(7)$.
 $P(2) = $
 \Solution $$P(2)\ =\ (20\ +\ 6\ +\ 3) \bmod 7\ =\ (6\ +\ 6\ +\ 3) \bmod 7\ =\ 15 \bmod 7\ =\ 1$$
 \end{Freeform}
 
\item What is the share given to the third official?
 \begin{Freeform}{1}
 % \Hint Remember we're working in $GF(7)$.
 $P(3) = $
 \Solution $$P(3)\ =\ (45\ +\ 9\ +\ 3) \bmod 7\ =\ (3\ +\ 2\ +\ 3) \bmod 7\ =\ 8 \bmod 7\ =\ 1$$
 \end{Freeform}
 
  \item What is the share given to the fourth official?
 \begin{Freeform}{4}
 % \Hint Remember we're working in $GF(7)$.
 $P(4) = $
 \Solution $$P(4)\ =\ (5 \times 16\ +\ 12\ +\ 3) \bmod 7\ =\ (3\ +\ 5\ +\ 3) \bmod 7\ =\ 11 \bmod 7\ =\ 4$$
 \end{Freeform}
 
  \item What is the share given to the fifth official?
 \begin{Freeform}{3}
 % \Hint Remember we're working in $GF(7)$.
 $P(5) = $
 \Solution $$P(5)\ =\ (5 \times 25\ +\ 15\ +\ 3) \bmod 7\ =\ (6\ +\ 1\ +\ 3) \bmod 7\ =\ 10 \bmod 7\ =\ 3$$
 \end{Freeform}
 
 \item Suppose officials $1$, $2$, and $5$ get together, and try to recover the secret. Using Lagrange interpolation, they compute their delta functions $\Delta_1(x), \Delta_2(x), \Delta_5(x)$. What are $a,b,c$ when $\Delta_1(x) = a(x-b)(x-c)$? Again note that $b < c$ and $b,c$ are integers.
  \begin{Freeform}{2}
 $a = $
 \Hint This should not be a fraction. Recall that we are in GF(7).
 \Solution $a = 2$. See solution to $c$ for explanation.
 \end{Freeform}
  \begin{Freeform}{2}
 $b = $
 \Solution $b = 2$. See solution to $c$ for explanation.
 \end{Freeform}
  \begin{Freeform}{5}
 $c = $

 \Solution First, we know we want $\Delta_1(x)$ to be $0$ when $x\, =\, 2$ and $x\, =\, 5$. This gives us 

 $$\Delta_1(x)\ =\ a(x - 2)(x - 5)\ \bmod 7$$

 Second, we want $\Delta_1(x)$ to be $1$ when $x\, =\, 1$. This will determine the value of $a$.

 $$\Delta_1(1)\ =\ a(1 - 2)(1 - 5)\ \bmod 7\ =\ 4 a \bmod 7\ =\ 1$$

 Now we know $a\, =\, 4^{-1} \bmod 7\ =\ 2$. This gives us $a\, =\, 2$, $b\, =\, 2$, $c\, =\, 5$
 \end{Freeform}
 
 \item What are $a,b,c$ when $\Delta_2(x) = a(x-b)(x-c)$? Again note that $b < c$ and $b,c$ are integers.
  \begin{Freeform}{2}
 $a = $
 \Hint This should not be a fraction. Recall that we are in GF(7).
 \Solution $a = 2$. See solution to $c$ for explanation.
 \end{Freeform}
  \begin{Freeform}{1}
 $b = $
 \Solution $b = 1$. See solution to $c$ for explanation.
 \end{Freeform}
  \begin{Freeform}{5}
 $c = $

 \Solution First, we know we want $\Delta_2(x)$ to be $0$ when $x\, =\, 1$ and $x\, =\, 5$. This gives us 

 $$\Delta_2(x)\ =\ a(x - 1)(x - 5)\ \bmod 7$$

 Second, we want $\Delta_2(x)$ to be $1$ when $x\, =\, 2$:

 $$\Delta_2(2)\ =\ a(2 - 1)(2 - 5)\ \bmod 7\ =\ 4a \bmod 7\ =\ 1$$

 Now we know $a\, =\, 4^{-1} \bmod 7\ =\ 2$. This gives $a\, =\, 2$, $b\, =\, 1$, $c\, =\, 5$
 \end{Freeform}
 
  \item What are $a,b,c$ when $\Delta_5(x) = a(x-b)(x-c)$? Again note that $b < c$ and $b,c$ are integers.
  \begin{Freeform}{3}
 $a = $
 \Hint This should not be a fraction. Recall that we are in GF(7).
 \Solution $a = 3$. See solution to $c$ for explanation.
 \end{Freeform}
  \begin{Freeform}{1}
 $b = $
 \Solution $b = 1$. See solution to $c$ for explanation.
 \end{Freeform}
  \begin{Freeform}{2}
 $c = $

 \Solution First, we know we want $\Delta_5(x)$ to be $0$ when $x\, =\, 1$ and $x\, =\, 2$. This gives us 

 $$\Delta_5(x)\ =\ a(x - 1)(x - 2)\ \bmod 7$$

 Second, we want $\Delta_5(x)$ to be $1$ when $x\, =\, 5$:

 $$\Delta_5(5)\ =\ a(5 - 1)(5 - 2)\ \bmod 7\ =\ 5a \bmod 7\ =\ 1$$

 Now we know $a\, =\, 5^{-1} \bmod 7\ =\ 3$. This gives us $a\, =\, 3$, $b\, =\, 1$, $c\, =\, 2$
 \end{Freeform}

 
 \item Their final polynomial $p(x) = a_2 x^2 + a_1 x + a_0 = P(1)\cdot\Delta_1(x) + P(2)\cdot\Delta_2(x) + P(5)\cdot\Delta_5(x)$ has coefficients:
  \begin{Freeform}{5}
 $a_2 = $
 \Hint Recall that we are in GF(7). This should be between $0$ and $6$.
 \Solution $a_2 = 5$. See solution to $a_0$ for explanation.
 \end{Freeform}
  \begin{Freeform}{3}
 $a_1 = $
 \Solution $a_1 = 3$. See solution to $a_0$ for explanation.
 % \Hint Recall that we are in GF(7). This should be between $0$ and $6$.
 \end{Freeform}
  \begin{Freeform}{3}
 $a_0 = $
 % \Hint Recall that we are in GF(7). This should be between $0$ and $6$.
 \Solution Let's write out the expression for $P(1)\Delta_1(x)\ +\ P(2)\Delta_2(x)\ +\ P(5)\Delta_5(x)$:

 $$4 \times 2(x - 2)(x - 5)\ +\ 1 \times 2(x - 1)(x - 5)\ +\ 3 \times 3(x - 1)(x - 2)\ \bmod 7$$

Simplifying, we have  

 \begin{align*}
 P(x)\ &=\ (x - 2)(x - 5)\ +\ 2(x - 1)(x - 5)\ +\ 2(x - 1)(x - 2)\ &\bmod 7\\
 &=\ x^2\ -\ 7x\ +\ 10\ +\ 2x^2\ -\ 12x\ +\ 10\ +\ 2x^2\ -\ 6x\ +\ 4\ &\bmod 7\\
 &= 5x^2\ -\ 11x\ +\ 10\ &\bmod 7\\
 &= 5x^2\ +\ 3x\ +\ 3\ &\bmod 7 
 \end{align*}

 This gives $a\, =\, 5$, $b\, =\, 3$, and $c\, =\, 3$. We have successfully recovered the original polynomial $P(x)$.

 \end{Freeform}
 
 \item Could officials $1$ and $2$ recover the secret with official $4$ instead of collaborating with official $5$?
 \begin{Choices}
\begin{itemize}
\TrueChoice\item Yes 
\FalseChoice\item No
\Solution Yes, any three points is enough to recover the polynomial.
\end{itemize}
\end{Choices}

 \item Could officials $1$ and $2$ recover the secret by only collaborating with each other?
 \begin{Choices}
\begin{itemize}
\FalseChoice\item Yes
\TrueChoice\item No
\Solution No, there are $7$ possible polynomials of degree $2$ with $P(1)$ and $P(2)$ fixed. Note that there were only seven possibilities for the original secret in the first place, so the last point contains just as much information as the secret itself.
\end{itemize}
\end{Choices}
 
\end{enumerate}

\end{enumerate}

\end{document}
