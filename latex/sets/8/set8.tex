% !TEX TS-program = pdflatex
% !TEX encoding = UTF-8 Unicode

% This is a simple template for a LaTeX document using the "article" class.
% See "book", "report", "letter" for other types of document.

\documentclass[11pt,preview]{standalone} % use larger type; default would be 10pt

\usepackage[utf8]{inputenc} % set input encoding (not needed with XeLaTeX)


\usepackage{../../markup}
%%% Examples of Article customizations
% These packages are optional, depending whether you want the features they provide.
% See the LaTeX Companion or other references for full information.

%%% PAGE DIMENSIONS
\usepackage{geometry} % to change the page dimensions
\geometry{a4paper} % or letterpaper (US) or a5paper or....
% \geometry{margin=2in} % for example, change the margins to 2 inches all round
% \geometry{landscape} % set up the page for landscape
%   read geometry.pdf for detailed page layout information

\usepackage{graphicx} % support the \includegraphics command and options
\usepackage{color, tikz}
% \usepackage[parfill]{parskip} % Activate to begin paragraphs with an empty line rather than an indent

%%% PACKAGES
\usepackage{amsmath, amsfonts,amssymb}
\usepackage{booktabs} % for much better looking tables
\usepackage{array} % for better arrays (eg matrices) in maths
\usepackage{paralist} % very flexible & customisable lists (eg. enumerate/itemize, etc.)
\usepackage{verbatim} % adds environment for commenting out blocks of text & for better verbatim
\usepackage{subfig} % make it possible to include more than one captioned figure/table in a single float
% These packages are all incorporated in the memoir class to one degree or another...

%%% HEADERS & FOOTERS
\usepackage{fancyhdr} % This should be set AFTER setting up the page geometry
\pagestyle{fancy} % options: empty , plain , fancy
\renewcommand{\headrulewidth}{0pt} % customise the layout...
\lhead{}\chead{}\rhead{}
\lfoot{}\cfoot{\thepage}\rfoot{}

%%% SECTION TITLE APPEARANCE
\usepackage{sectsty}
\allsectionsfont{\sffamily\mdseries\upshape} % (See the fntguide.pdf for font help)
% (This matches ConTeXt defaults)

%%% ToC (table of contents) APPEARANCE
\usepackage[nottoc,notlof,notlot]{tocbibind} % Put the bibliography in the ToC
\usepackage[titles,subfigure]{tocloft} % Alter the style of the Table of Contents
\renewcommand{\cftsecfont}{\rmfamily\mdseries\upshape}
\renewcommand{\cftsecpagefont}{\rmfamily\mdseries\upshape} % No bold!

\newcommand{\N}{\mathbb{N}}
\newcommand{\Z}{\mathbb{Z}}
\newcommand{\R}{\mathbb{R}}
\newcommand{\Q}{\mathbb{Q}}
%%% END Article customizations

%%% The "real" document content comes below...

\date{} % Activate to display a given date or no date (if empty),
         % otherwise the current date is printed 

\begin{document}
\config{name}{Counting}
\noindent{\bf Counting}.


\begin{enumerate}
\item Say I have a standard 6-sided dice. I generate a sequence of 5 numbers by tossing this dice 5 times, and writing down the result after each toss. 
\begin{enumerate}
\item How many possible distinct sequences of 5 numbers can I generate this way? Please enter your answer as an integer.
\begin{Freeform}{7776}
\# sequences = 
\Hint How many options do we have each time? Is repetition allowed? (Review the flipping coins and rolling dice sections of the notes).
\Solution The answer is $6\times 6\times 6\times 6\times 6= 7776$ because each time we have $6$ options. 
\end{Freeform}
%----------------------------------
\item True or False: In the above procedure, the sequence of numbers $6, 6, 6, 6, 6$ is less likely to come up that the sequence $6, 5, 6, 6, 5$.
\begin{Choices}
\Hint How many possible rolling outcomes generate the first outcome? How many possible rolling routines generate the second?
\begin{itemize}
\FalseChoice\item True 
\TrueChoice\item False
\end{itemize}
\Solution False. Both are equally likely to come up.
\end{Choices}
%----------------------------------
\item True or False: In the above procedure, the sequence of numbers $6, 6, 6, 6, 6$ is less likely to come up than some sequence of the form $x_1, x_2, x_3, x_4, x_5$, where each $x_i \in \{5,6\}$. 
\begin{Choices}
\Hint How many possible tosses generate the first outcome? How many possible tosses generate the second?
\begin{itemize}
\TrueChoice\item True 
\FalseChoice\item False
\end{itemize}
\Solution True. There is one way for the first sequence to come up. But there are many more ways for a sequence of the second form come up ($2^5$ ways).
\end{Choices}
%----------------------------------
\item How many sequences generated by the procedure above have the form $6,6,6,6,6$? Please enter your answer as an integer. 
\begin{Freeform}{1}
\# sequences = 
\Hint Review the flipping coins and rolling dice sections of the notes.
\Solution There is just one way, since every dice roll is already uniquely determined.
\end{Freeform}
%----------------------------------
\item How many sequences generated by the procedure above are of the form $x_1, x_2, x_3, x_4, x_5$, where each $x_i \in \{5,6\}$? Please enter your answer as an integer. 
\begin{Freeform}{32}
\# sequences = 
\Hint Review the flipping coins and rolling dice sections of the notes.
\Solution 32. For each roll we have 2 choices, so the answer is $2^5=32$.
\end{Freeform}
%----------------------------------
\item What is the probability of my generating the sequence $6,6,6,6,6$? Please enter your answer as a decimal with a leading zero and to exactly 5 decimal place precision  (round to the nearest $10^{-5}$, rounding up on 5).
\begin{Freeform}{0.00013}
probability = 
\Hint Recall that the definition of the probability of an event is the number of times this event occurs divided by the number of possible events. 
\Solution We divide the number of ways, i.e. $1$, by $6^5$ which is the total number of possible outcomes to get the answer.
\end{Freeform}
%----------------------------------
\item What is the probability of my generating the sequence $6, 5, 6, 6, 5$? Please enter your answer as a decimal with a leading zero and to exactly 5 decimal place precision  (round to the nearest $10^{-5}$, rounding up on 5).
\begin{Freeform}{0.00013}
probability = 
\Hint Recall that the definition of the probability of an event is the number of times this event occurs divided by the number of possible events.
\Solution As before, the number of ways, i.e. $1$, divided by the total number of outcomes, i.e. $6^5$, gives us the answer. 
\end{Freeform}
%----------------------------------
\item What is the probability of my generating a sequenceof the form $x_1, x_2, x_3, x_4, x_5$, where each $x_i \in \{5,6\}$? Please enter your answer as a decimal with a leading zero and to exactly 5 decimal place precision (round to the nearest $10^{-5}$, rounding up on 5).
\begin{Freeform}{0.00412}
probability = 
\Hint Recall that the definition of the probability of an event is the number of times this event occurs divided by the number of possible events. 
\Solution The total number of ways to get such a sequence is $2^5$. Dividing by $6^5$ gives us the probability.
\end{Freeform}
%----------------------------------
\end{enumerate}
\item Suppose that my 5 friends are roommates, and that they also share a sock drawer. In this sock drawer they have 10 pairs of socks. Each pair of socks is a different color, but it is folded together so that no roommate is ever wearing two different colors of socks.
\begin{enumerate}
\item How many distinct roommate-sock combinations are there? Please enter your answer as an integer. 
\begin{Freeform}{30240}
\# combinations = 
\Hint Review the counting sequences section of the notes.
\Solution For the socks of the first roommate, we have $10$ choices, for the second one, we have $9$ choices and so on. The answer is $10\times 9\times 8\times 7\times 6=30240$.
\end{Freeform}
%----------------------------------
\item Each day, the roommates record the combination of sock colors that they are all wearing in a special scrapbook. How many such combinations are possible?  Please enter your answer as an integer. 
\begin{Freeform}{252}
\# combinations = 
\Hint Notice that order doesn't matter anymore--we only care about which socks are present, not about which roommate is wearing them. Review the counting sets section of the notes.
\Solution The answer is $\binom{10}{5}=252$ which is the number of ways we can pick $5$ out of the $10$ socks when the order does not matter. 
\end{Freeform}
%----------------------------------
\item Assuming that the roommates randomly choose their socks each morning, what is the probability that the red, teal, orange, pink, and magenta socks are all worn by the roommates on a specific day? Please enter your answer as a decimal with a leading zero and to exactly 5 decimal place precision  (round to the nearest $10^{-5}$, rounding up on 5).
\begin{Freeform}{0.00397}
probability = 
\Hint Recall the definition of probability. What is the total number of events possible here?
\Solution There are $\binom{10}{5}$ possible combinations of socks, all of which are equally likely. Here we are interested in only one of these combinations. The probability of it is simply $1/\binom{10}{5}$.
\end{Freeform}
%----------------------------------
\item Say that one roommate decides he likes the pink, red, and teal socks more than all of the rest, and decides to jealously hoard them so that no one else can ever wear them (but he can still also wear the other 7 colors of socks). Now, how many distinct roommate-sock combinations are there? Please enter your answer as an integer. 
\begin{Freeform}{5040}
\# combinations = 
\Hint Review the counting sequences section of the notes. How many choices does each roommate get if the first roommate chooses a hoarded sock? How many choices does each roommate get if the first roommate chooses a shared sock?
\Solution Let's begin with the other roommates choosing their socks. The first one has $7$ choices, the second one has $6$, the third one $5$, and the fourth one $4$. Now the roommate who has hoarded socks can pick any pair from the remaining $3$ hoarded socks and $3$ not-hoarded ones. So this roommate has $6$ choices. The answer is $7\times 6\times 5\times 4\times 6=5040$.
\end{Freeform}
%----------------------------------
\item The other roommates grow frustrated with the hoarding roommate and tell him that if he does not return the pink, red, and teal socks, then he is not allowed to wear the other colors of socks. The hoarding roommate does not relent. Now, how many distinct roommate-sock combinations are there? Please enter your answer as an integer. 
\begin{Freeform}{2520}
\# combinations = 
\Hint Review the counting sequences section of the notes. How many choices does each roommate get?
\Solution The hoarding roommate has $3$ choices. The next roommate has $7$ (all the not-hoarded socks), the next one $6$, the next $5$, and the next $4$. So the answer is $7\times 6\time 5\times 4\times 3=2520$.
\end{Freeform}
%----------------------------------
\item How many color combinations of socks are now possible in the scrapbook? Please enter your answer as an integer. 
\begin{Freeform}{105}
\# combinations = 
\Hint Review the counting sets section of the notes. Which sets are being chosen from?
\Solution We need to pick $4$ colors from the $7$ not-hoarded socks and one from the hoarded socks. So the answer is $\binom{7}{4}\times\binom{3}{1}=105$.
\end{Freeform}
\end{enumerate}

% ---------------------------------------------------------------
\item How many ways are there to arrange the letters of the word "SUPERMAN"
\begin{enumerate}
\item  On a straight line? 
\begin{Freeform}{40320}
\end{Freeform}

\item On a straight line, such that "SUPER" occurs as a substring?
\begin{Freeform}{24}
\end{Freeform}

\item On a circle? Note: If we arrange elements on a circle, all permutations that are ``shifts'' are equivalent (i.e. SUPERMAN and UPERMANS).
\begin{Freeform}{5040}
\Solution Anchor one element, arrange the other 7 around in a line.
\end{Freeform}

\item On a circle, such that "SUPER" occurs as a substring? Reminder: SUPER can occur anywhere on the circle!
\begin{Freeform}{6}
\Solution Treat "SUPER" as a single character, anchor one element, and arrange the other 3 around in 
a line.
\end{Freeform}

\item On a straight line, such that "SUPER" occurs as a subsequence 
(S U P E R appear in that order, but not necessarily next to each other)? 
\begin{Freeform}{336}
\Solution There are two ways to think about the problem.

We can arrange the letters of SUPERMAN 8! ways, but 
divide by 5! because we have arranged SUPER in any of 5! ways, when 
we only want one way. This gives us 8! / 5!.

Alternatively, we can think about picking three slots for "MAN" and then permute them. Then "SUPER" should fill in the other 5 slots with S going first and then U, P, E, R as they have to appear as a subsequence. This gives us 3! * ${8 \choose 3}$.

You're encouraged to check that the above two methods give the same answer. 
\end{Freeform}

\item On a circle, such that "SUPER" occurs as a subsequence (S U P E R 
appear in that order, but not necessarily next to each other)? 
\begin{Freeform}{210}
 \Solution ${7 \choose 3} * 3!$ = ${7 \choose 2} * 2! * 5= 210$. 

 There are two methods. Method 1: anchor one of {S, U, P, E, R}. Choose which 3 places to put the M, A, and N (7 choose 3) and allow them to be shuffled (3!). Then the U, P, E, R must fill in the remaining slots in order. We get  ${7 \choose 3} * 3!$.

 Method 2: anchor one of {M, A, N}. Choose which of the 2 remaining 8 spots to place the A and N, allowing shuffles (7 choose 2 * 2!). Then, choose which of the 5 remaining spots to place the S (the other letters must follow in order after the S). We get ${7 \choose 2} * 2! * 5$.

 You are encouraged to check the two answers above are equivalent.
\end{Freeform}
\end{enumerate}

% ---------------------------------------------------------------------
\item How many ways can you give 10 cookies to 4 friends?
\begin{Freeform}{286}
\Solution Count the number of ways to give 10 cookies to 4 friends if 
some can get no cookies. Using the model of stars and bars,
we have $10$ stars and $4 - 1 = 3$ bars. The number of ways to
arrange them is ${13 \choose 3}  = 286$. 
\end{Freeform}

%-----------------------------------------------------------------------

\item How many 5-digit sequences have the digits in ...
\begin{enumerate}
\item strictly increasing order?
\begin{Freeform}{252}
\Solution This is equivalent to choosing five digits without replacement and order doesn't matter, which corresponds to ${10} \choose {5}$ ways. 
\end{Freeform}
\item non-decreasing order?
\begin{Freeform}{2002}
\Solution This can be framed as a stars and bars problem. We have 9 bars representing the separation of 10 types of digits (0, 1, ..., 9) and 5 stars representing the 5 digits. The location of a star represents the value of its associated digit. For example, a star placed before the first bar represents 0, between the first bar and second bar represents 1, etc. There are ${14 \choose 9} = 2002$ ways to arrange 9 bars and 10 stars, which gives us the answer. 
\end{Freeform}
\end{enumerate}
% ----------------------------------------------------------------------------

\item How many solutions does $x + y + z = 10$ have, if all variables 
must be positive integers? 
\begin{Freeform}{36}
\Solution We can think of this in terms of stars and bars. We have two bars between the variables x, y, and z, and our stars are the 10 1s we have to distribute among them. Since all variables must be positive integers, x, y, and z will each be at least 1. So, we have 7 1s left to distribute. So we have $7$ stars, $2$ bars. Answer = $ {9 \choose 2} = 36$.
\end{Freeform}

\item A combinatorial proof is a proof which shows that two quantities are the same by explaining that each quantity is a different way of counting the same thing. This question is intended to help you see how this technique is applied.

In the question below, a quantity is given. Select all choices that are a valid different way of counting the given quantity (allowing you to deduce that the formulae given by each way of counting are equal).
\begin{enumerate}
\item The number of squares in an $n \times n$ grid.
\begin{enumerate}
\begin{Multi}
\TrueChoice\item In an $n \times n$ grid, there are $n$ rows of squares, each of which has $n$ squares in it. Thus, there are $n^2$ squares in an $n \times n$ grid. 
\TrueChoice\item We know there are exactly $n$ squares on the diagonal. Now, when we remove the diagonal, we have two equally sized triangles that have $n-1$ squares on the hypotenuse. When we remove those, we end up with smaller triangles with $n-2$ squares on the hypothenuse. We continue this until we are left with one square on each side, and we've counted all of the squares in the grid. This gives us a total of $n + 2\sum_{k = 1}^{n-1} k$ squares in the grid. 
\TrueChoice\item Take the $(n-1) \times (n-1)$ subgrid that is the upper lefthand corner of this grid. This subgrid has $n-1$ rows, each of which has $n-1$ squares, so this part contributes $(n-1)^2$ squares. Now, the squares that we excluded from this subgrid come to a total of $n + n - 1$ squares. Thus, there are $(n-1)^2 + 2n -1$ squares in an $n \times n$ grid.
\TrueChoice\item First, we peel off the leftmost column, and topmost row, removing exactly $2n - 1$ squares. We then peel off the leftmost column and topmost row remaining, removing exactly $2(n-1) - 1$ squares. We continue this process until we are left with a single square, which we also remove. This gives us a total of $(2n - 1) + (2n - 3) + \cdots + 3 + 1 = \sum_{k=1}^{n} 2k - 1$ squares in the $n\times n$ grid. 
\Hint Review combinatorial proofs in the notes. Also, drawing out a couple of examples might help.  
\Solution All of these are valid ways of counting the number of squares in an $n\times n$ grid.
\end{Multi}
\end{enumerate}
%------------------------------------------------------------------------------
\end{enumerate}
Notice you can verify that these quantities are actually equal algebraically. While this example was simple enough to do so, the proof technique also extends to situations where the equality is less obvious. 

For example, you can try to prove that $\binom{n}{k} = \frac{n!}{k!(n-k)!}$ by combinatorially proving that \[
\binom{n}{k}k! = n(n-1) \cdots (n-k+1),\] 
using only the set-counting definition of $\binom{n}{k}$. What is the quantity on the left counting? Can you argue that the quantity on the right is counting the same thing?

% -----------------------------------------------------------------------------------------
\item Fill in the blank:
$k\binom{n}{k} = \cdots$, 

\begin{Choices}
\Hint Consider picking a team and a captain.
\begin{enumerate}
\FalseChoice\item $(n - 1)\binom{n-1}{k}$
\TrueChoice\item $n\binom{n - 1}{k - 1}$
\FalseChoice\item $n\binom{n - 1}{k}$
\FalseChoice\item $\binom{n}{k+1}$

\Solution Choose a team of $k$ players where one of the players is the captain. \\
LHS: Pick a team with $k$ players. This is ${n \choose k}$. Then make one of the players the captain. There are $k$ options for the captain so we get $k \times \binom{n}{k}$. \\
RHS: Pick the captain. There are $n$ choices for the captain. Now pick the last $k - 1$ players on the team. There are now $n - 1$ people to choose from. So we get $n \times \binom{n - 1}{k - 1}$.
\end{enumerate}
\end{Choices}

% ------------------------------------------------------------------------------------------
\item Fill in the blank:
$n! = \cdots$
\begin{enumerate}
\begin{Choices}
\FalseChoice\item $\binom{n - k}{k}k!(n - k)!$
\FalseChoice\item $\binom{n}{k}k!$
\TrueChoice\item $\binom{n}{k}k!(n - k)!$
\FalseChoice\item $\binom{n}{k}(n-k)!$
\Solution Arrange $n$ items. \\
LHS: Number of ways to order $n$ items. \\
RHS: Choose $k$ items without ordering. Order these $k$ items. 
Order the remaining $n - k$ items.
\end{Choices}
\end{enumerate}

% ---------------------------------------------------------------------------------------------
\item Fill in the blank:
$\binom{n}{a} a(n - a)= \cdots$ 

\begin{Multi}
\Hint Consider ways of picking a team, a captain and a reserve player. 
\begin{enumerate}

\TrueChoice\item $n(n - 1)\binom{n - 2}{a - 1}$
\FalseChoice\item $n \binom{n - 1}{a - 1}$
\FalseChoice\item $\binom{n}{a+1}$
\FalseChoice\item $(a+1)\binom{n}{a+1}$
\TrueChoice\item $(a+1)a \binom{n}{a+1}$
\Solution Suppose that you have a group of $n$ players and want to pick a team of $a$ with a captain, as well as a reserve player from the remaining $n-a$ players. \\
LHS: Number of ways to pick a team of $a$ of these players ($\binom{n}{a} $ ways), designate 
one member of the team as captain ($a$ ways), and then pick one reserve player 
from the remaining $n - a$ people ($n-a$ ways). \\
RHS: The right-hand side is the number of ways to pick the captain ($n$ ways), then the reserve player ($n-1$ ways), and then the 
other $a - 1$ members of the team ($\binom{n - 2}{a - 1}$ ways).
\end{enumerate}
\end{Multi}



\end{enumerate}
%--------------------------------------------------------------------------------------------------------------------------


\end{document}
